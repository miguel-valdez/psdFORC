\documentclass[review,authoryear]{elsarticle}
%-----------------------------------------------------

\usepackage{natbib}
\usepackage{amsmath}
\usepackage{graphicx}
\usepackage{url}
\usepackage{lineno}
\usepackage{caption}

\journal{Earth and Planetary Science Letters}

\bibliographystyle{elsarticle-harv}
\setcitestyle{authoryear}

\graphicspath{{./figs/}}

\newcommand{\roughly}{{\raise.17ex\hbox{$\scriptstyle\sim$}}}
\newcommand{\nm}{\,\text{nm}}
\newcommand{\mT}{\,\text{mT}}
%-----------------------------------------------------

\begin{document}
\allowdisplaybreaks

\begin{frontmatter}

\title{Pseudo single-domain effects on the first order reversal curve (FOR) diagram of greigite dispersions}

\author[ic]{Miguel~A.~Valdez-Grijalva\corref{co}}
\ead{m.valdez-grijalva13@imperial.ac.uk}
\cortext[co]{Corresponding author}

\author[ic]{Adrian~R.~Muxworthy}
\author[ed]{Wyn~Williams}
\author[ed]{Lesleis~Nagy}
\author[ed]{P{\'a}draig~{\'O}~Conbhu{\'i}}
\author[au]{Andrew~P.~Roberts}

\address[ic]{Department of Earth Science and Engineering, Imperial College London, SW7 2BP, UK}
\address[ed]{School of GeoSciences, University of Edinburgh, EH9 3FE, UK}
\address[au]{Research School of Earth Sciences, Australian National University, 2601, Australia}

\begin{abstract}
First-order reversal curve (FORC) diagrams have increasingly become an important and widespread technique in rock magnetic studies. The FORC signal of fine, single-domain (SD) particles is well studied. However, the SD nature of these studies limits the applicability to many systems where it is known that the magnetic signal is dominated by particles with non-uniform magnetisations. In this study, the properties of noninteracting greigite (Fe$_3$S$_4$) dispersions in the single-domain (SD) to single-vortex (SV) size range are investigated via micromagnetic calculations. The SD signal was found to be in excellent agreement with previous SD coherent-rotation studies. The FORC diagrams were found to be highly sensitive to the magnetic domain state of the particles. A transitional range from $\roughly$50$\nm$ to $\roughly$70$\nm$ was identified for which a mixture of SD and SV behaviour produce complex FORC diagrams. Particles larger than $\roughly$70$\nm$ show a purely SV behaviour with the remanent state for all particles in the ensemble a vortex state. Classic interpretations of the FORC diagram hold for SD ensembles whereas, for SV ensembles the FORC distribution maximum should not be interpreted as the coercivity of the sample but as the field intensity necessary to unravel the magnetic vortex structure of these particles.
\end{abstract}

\begin{keyword}
Greigite \sep Pseudo Single-Domain \sep Micromagnetics \sep FORC diagram
\PACS 91.25.F- \sep 91.60.Pn \sep 91.25.fa
\end{keyword}

\end{frontmatter}

\linenumbers
%-----------------

\section{Introduction}

First-order reversal curve (FORC) diagrams are a powerful tool in rock magnetic studies which allow for mineral and domain state identification as well as quantification of magnetostatic interactions between particles \citep{Roberts2000,Roberts2014}. As such, they have been the subject of numerical studies aiming to relate the individual behaviour of the magnetic particles and small assemblages to the experimental bulk properties \citep{Carvallo2003,Carvallo2006,Muxworthy2004,Muxworthy2005,Newell2005,Harrison2014}.\par

%The FORC diagram properties of single-domain (SD) particles have been studied in both the noninteracting \citep{Newell2005} and interacting cases \citep{Muxworthy2004,Muxworthy2005}. However, iron-rich minerals of the utmost importance to the Earth Sciences like magnetite (Fe$_3$O$_4$) and greigite (Fe$_3$S$_4$) do not posses a uniaxial MCA but rather, cubic. Also, these studies do not take into account size effects as they effectively treat each particle as a single magnetic dipole.\par

The noninteracting FORC signal of SD particles with cubic magnetocrystalline anisotropy (MCA) was recently calculated by \citet{ValdezGrijalva2017} using a simple dipole model. Whereas, \citet{Carvallo2003,Carvallo2006,Muxworthy2004,Muxworthy2005} employed numerical micromagnetic models to study the effects of magnetostatic interactions in small assemblages of SD magnetite. However, the FORC diagram of particles with nonuniform magnetisations, specially those in a single-vortex (SV) state, is not yet fully understood.\par

In this study, we employ a micromagnetic finite element method (FEM) to study the FORC diagram properties of noninteracting ensembles of greigite and determine the onset of pseudo single-domain (PSD) behaviour. The magnetic parameters of greigite used in this investigation are: the saturation magnetisation $M_\text{S}=3.51\,\mu_\text{B}\,\text{p.c.u.}$ or $\roughly 2.7 \times 10^5\,\text{A/m}$ \citep{Guowei2014}; the exchange stiffness constant $A=2\times10^{-12}\,\text{J}/\text{m}$ \citep{Chang2008}; and the first MCA constant $K_1=-1.7\times10^4\,\text{J}/\text{m}^3$ \citep{Winklhofer2014}. This set of parameters has been used in a recent micromagnetic study of greigite \citep{ValdezGrijalva2017B}.\par
%-----------------------------------------------------

\section{Methodology}
\subsection{The micromagnetic algorithm}
A ferromagnetic material{\textemdash}neglecting thermal and magnetostrictive effects{\textemdash}has a Gibbs free-energy functional given by \citep{Brown}
{\par\nobreak\noindent}
\begin{equation}
E_\text{G} = \int_{\Omega} (\phi_{\text{exchange}} + \phi_{\text{anisotropy}} + \phi_{\text{stray}} + \phi_{\text{external}})\,\text{d}^3 \boldsymbol{r},
\end{equation}
with $\Omega$ the ferromagnetic volume. Here,
{\par\nobreak\noindent}
\begin{equation}
\phi_{\text{exchange}}=A|\nabla\boldsymbol{m}|^2
\end{equation}
 with the reduced magnetisation vector $\boldsymbol{m}$, is the expression for the energy density due to the quantum-mechanical exchange forces \citep{Landau1935}.
{\par\nobreak\noindent}
\begin{equation}
\phi_{\text{anisotropy}}=\frac{K_1}{2}\sum_{i\neq j}\gamma_i^2\gamma_j^2 + K_2\prod_i\gamma_i^2,
\end{equation}
with $\gamma_i$ the direction cosines, is the MCA energy density in the cubic anisotropy system. In terms of the reduced magnetisation vector components this becomes:
{\par\nobreak\noindent}
\begin{equation}
\phi_{\text{anisotropy}}=K_1(m_x^2m_y^2+m_y^2m_z^2+m_z^2m_x^2),
\end{equation}
where $K_2$ has been neglected since $K_1$ is the dominant term at room temperature. The magnetostatic self-energy density is given by
{\par\nobreak\noindent}
\begin{equation}
\phi_{\text{stray}}=-\frac{\mu_0M_\text{S}}{2}\boldsymbol{m}\cdot\boldsymbol{H}_{\text{stray}},
\end{equation}
with $\boldsymbol{H}_{\text{stray}}$ the stray field produced by the ferromagnetic body. Finally, the energy density due to an external magnetic field $\boldsymbol{H}_{\text{external}}$ is
{\par\nobreak\noindent}
\begin{equation}
\phi_{\text{external}}=-\mu_0M_\text{S}\boldsymbol{m}\cdot\boldsymbol{H}_{\text{external}}.
\end{equation}\par

It is known that the system will be spontaneously driven towards an equilibrium state with a locally minimal magnetic Gibbs free-energy \citep{Brown}. In this study we utilise a modified gradient descent method with the aim of finding the equilibrium magnetisation \citep{OConbhui2017}.\par

The discretisation of the spatial domain is achieved by a decomposition of the volume into tetrahedral elements. This allows for arbitrary geometries to be modelled. To model accurately nonuniform magnetisations the spatial discretisation in the model should be smaller than the exchange length $l_\text{exch.} = \sqrt{2A/\mu_0M_\text{S}^2}$ \citep{Rave1998}, which for greigite is $l_\text{exch.} \approx 6.6\, \text{nm}$; a maximum element size of 5$\nm$ has been chosen for all the models. The non-local problem of calculating the stray field is resolved by a hybrid finite-element/boundary-element formulation \citep{Fredkin1990}.\par

\subsection{The FORC model}
FORC diagrams are constructed from a class of partial hysteresis curves called first order reversal curves, each starting at some value of the applied field $B_a$ along the main hysteresis branch. A magnetisation function on two variables $M=M(H_a, H_b)$ is thus obtained. The FORC distribution $\rho$ is then defined as \citep{Roberts2000}:
{\par\nobreak\noindent}
\begin{equation}
\rho=-\frac{1}{2}\frac{\partial^2 M}{\partial H_a \partial H_b}.
\end{equation}\par

Once $M(H_a, H_b)$ is obtained, the calculation of $\rho(H_a, H_b)$ is done by least-square fitting a degree 2 polynomial surface $a_0 + a_1 H_a + a_2 H_b + a_3 H_a H_b + a_4 H_a^2 + a_5 H_b^2 = 0$ on a subgrid of $M(H_a, H_b)$ centered around $H_a, H_b$ determined by the so-called smoothing factor (SF) and including (2$\times$SF + 1)$^2$ points; the value of $\rho$ is then simply $-a_3/2$ \citep{Pike1999}. Contour plots of the FORC distribution are called FORC diagrams. Usually they are presented in the rotated axes $H_c=(H_b - H_a)/2$, $B_u=(H_b + H_a)/2$.\par

Random field orientations from a distribution uniform over a sector of the sphere have been chosen (Fig. \ref{FIG_01}). We have chosen to use 500 field orientations as a good compromise between accuracy and calculation speed (Fig. \ref{FIG_02}). Also, like \citet{ValdezGrijalva2017} we note that for each particle/field-orientation, we only need to calculate the main branch of the hysteresis loop and the few reversal curves starting at the different switching fields along the main branch. These simplifications vastly reduce the amount of calculations needed without loss of generality. The external field rate of change for all models was 1$\mT$ with a saturation field of 250$\mT$ thus, 501 reversal curves were calculated for each particle/field-orientation.\par

Scanning electron microscopy and transmission electron microscopy micrographs of naturally occurring greigite samples \citep{Snowball1997,Vasiliev2008,Roberts2015} reveal that greigite tends to grow authigenically as well-defined regular truncated octahedral particles. Micromagnetic calculations for truncated octahedral greigite particles show the SD--PSD threshold to be $\roughly 53\nm$ \citep{ValdezGrijalva2017B}. In this study we model the FORC diagram of noninteracting ensembles of greigite particles sized 30$\nm$--80$\nm$ (size normalised to the volume of a cube) every 2$\nm$ as this range covers the transition from SD to SV behaviour.\par

%-----------------------------------------------------

\section{Results and Discussion}
A comparison with the dipolar, coherent rotation greigite model of \citet{ValdezGrijalva2017} shows excellent agreement (Fig. \ref{FIG_02}) for all particles up to 48$\nm$, with their FORC diagrams showing the same pattern. This means that in ensembles with grains smaller than 50$\nm$ the hysteretic behaviour is dominated by purely SD particles with coherent rotations (Fig. \ref{FIG_02}). The diagrams obtained with the micromagnetic algorithm are offset slightly towards the left compared to the dipolar model; this can be attributed to the micromagnetic algorithm including self-interaction effects as well as flowering (small deviations from a perfect SD structure). The FORC diagram of the SD coherent-rotating particles shows the same general features as those obtained for weakly interacting SD particles with cubic MCA by \citet{Harrison2014}, i.e., a positive ridge along the $H_c$ axis slightly offset towards negative $H_u$ values and a negative ridge at $\roughly$45 degrees from the lower part of the positive one.\par

For particles with size $d> 50\nm$ the FORC diagrams begin to depart from coherent rotation SD-like behaviour as the tight boomerang-shaped FORC diagram pattern exhibited by the SD greigite becomes more fragmented (Figs. \ref{FIG_03}a--f). This change is driven initially by the particles with a hard axis close to the applied field nucleating hard-aligned vortices (HAVs) during rotation for negative values of the applied field. Even though the nucleation of HAVs is happening for particles with sizes below the zero-field SD--PSD threshold $d_0=\roughly$53$\nm$ \citep{ValdezGrijalva2017B}, this is expected as the nucleation of a vortex greatly reduces the Gibbs free-energy due to the external and demagnetising (stray) fields. Fragmentation of the FORC diagram for non-uniformly magnetised particles was also observed by \citet{Carvallo2003}; however, they included magnetostatic interactions between the particles and grain elongations. The trend is nevertheless clear and may be representative of the complex self-interactions brought about by the nonuniform structures.\par

Above 56$\nm$ (Figs. \ref{FIG_03}d--f) an appreciable positive source in the FORC distribution appears along the $B_u=\mu_0 H_u=0$ axis at $B_c=\mu_0 H_c=\roughly 52\mT$; this contribution represents the unravelling of vortex states on the way to saturation. The elongated, negative ridge of the SD diagram moves towards lower values of $B_c, B_u$ and the first sources for $B_u > 0$ begin to form; these are elongated features at a 45 degree angle from the $B_u=0$ axis, different from the vertical widening usually attributed to the presence of magnetostatic interactions \citep{Muxworthy2004,Muxworthy2005}.\par

For these particles slighly above the SD--PSD threshold $d_0$, vortex nucleation occurs for negative values of the applied field, thus the change in the FORC diagram is not reflected in a change in the remanence up to 72$\nm$, whereas the coercivities sharply decrease above 48$\nm$ (Fig. \ref{FIG_04}b). The monotonically-decreasing trend of the coercivites is preserved up to 62$\nm$ when the coercivity rises from $B_{\text{C}}=\roughly 15 \mT$ to $B_{\text{C}}=\roughly 20 \mT$ (coercivity $B_{\text{C}}$ not to be confused with $B_c$) for $d=68\nm$. On increasing size, the coercivities further decrease accompanied by a sharp decrease in the remanence (Fig. \ref{FIG_04}b). The drop in the remanence values is driven by particles nucleating vortices at $B_a>0$ for $d \geq 68\nm$. For $d \geq 76\nm$, all particles nucleate vortices which become the remanent magnetic domain state. This is reflected in the Day plot (Fig. \ref{FIG_04}a) by the particles sized 76$\nm$ and larger plotting in the region usually designated as PSD.\par

Particles sized 62$\nm$ to 72$\nm$ move away from the Day plot region usually attributed to SD grains (Fig. \ref{FIG_04}a) to a region with high remanence but larger values of the coercivity of remanence $B_{\text{CR}}$ to $B_{\text{C}}$ ratio. These sizes coincide with the anomalous behaviour of the coercivity rising for these sizes (\ref{FIG_04}b). The FORC diagrams for these sizes are the most complex of all, showing a variety of X-shaped features and the appearance of a substantial symmetric negative source at $B_c=\roughly$5$\mT$, $B_u=\roughly$-10$\mT$ (Fig. \ref{FIG_05}). The elongated, negative ridge becomes more faint with size. At the same time, the positive sources for $B_u>0$ become stronger and moving towards the $B_c=0$ axis with size. Strong, positive FORC distribution sources for $B_u>0$ along the $B_c=0$ axis are expected for the larger MD grains \citep{Roberts2006}, and this behaviour might be representative of that tendency.\par

The elongated, negative ridge typical of cubic MCA SD particles \citep{ValdezGrijalva2017} disappears for the particles sized 74$\nm$ and larger (Fig. \ref{FIG_06}). The symmetric negative feature near the origin becomes stronger and of a magnitude comparable to the largest positive source. For $d=76\nm$ and 78$\nm$ the negative source is larger than the highest positive one. Towards 80$\nm$ a faint negative source appears close to the positive source situated along the $B_u=0$ axis. These diagrams represent the contribution of purely SV particles, that is, ensembles of particles that are all in a vortex remanent state. It is logical that these are somewhat less complex than the diagrams of ensembles comprised of a fraction of particles still in the SD state as well as SV.\par

An averaged-over-size diagram was obtained for a flat particle size distribution between 60$\nm$ and 80$\nm$ (Fig. \ref{FIG_07}). This SV-dominant diagram has a significative spread towards positive $B_u$. This effect is purely due to domain state, not magnetostatic interactions. The maximum for the SD-dominant diagrams occurs along the $B_u=0$ axis at $B_c=\roughly$25$\mT$, the value of the ensemble coercivity; whereas, for the SV-dominant diagrams this is at $B_c=\roughly$52$\mT$ and thus not related to the coercivity of the ensemble. This is a departure from the usual interpretation of FORC diagram, i.e., that the FORC diagram maps the coercivities. This interpretation holds for the SD coherent-rotation grains, as this source coincides with the value of the ensemble coercivity. It does not hold, however, for the SV grains as their coercivity decreases with size while the position of the maximum moves towards higher values of $B_c$. Instead, for SV grains this maximum should be interpreted as the applied field necessary to unravel the vortices.\par


%-----------------------------------------------------

\section{Conclusion}
A micromagnetic FEM was employed for the calculation of the FORC diagrams of noninteracting ensembles of greigite across a size range that spans the SD to PSD threshold. 500 random orientations from a distribution uniform over a sector of the sphere were used for each size. This choice was found to be in excellent agreement with previous calculations \citep{ValdezGrijalva2017} for coherent-rotating SD particles using a regular grid of field orientations (Fig. \ref{FIG_02}).\par

The FORC diagrams showed to be extremely sensitive to the domain state of the particles. As soon as a fraction of particles starts nucleating vortices, at $\roughly$50$\nm$, this is reflected in the FORC diagram. The same cannot be said of the Day plot which `treats' the particles up to 60$\nm$ as SD.\par

Anomalous behaviour for particles sized 62$\nm$ to 72$\nm$ consisting in an increase of the coercivity with size was found; these particles plot in an unexpected region of the Day plot. The anomaly disappears for particles $>72\nm$, and when $>76\nm$ they appear in the region usually attributed to PSD grains.\par

Finally, it is possible that previous notions on how the FORC diagrams should be interpreted need to be updated. This is illustrated by the disconnect on what the FORC distribution maxima mean for SD or PSD particles. For SD particles, the typical interpretation of the maximum position coinciding with the coercivity of the sample holds. However, for SV-dominated samples, the position of the maximum occurs at a value much higher than the coercivity of the sample and is actually related to the field intensity necessary to unravel the vortices formed as the remanent states.\par
%-----------------------------------------------------

\section*{Acknowledgments}
This research was partially funded by Instituto Mexicano del Petr\'oleo (M. A. Valdez-Grijalva) as well as by NERC grant NE/J020508/1 (A. R. Muxworthy and W. Williams).\par
%-----------------------------------------------------

\def\urlprefix{}
\bibliography{references}
\newpage
%-----------------------------------------------------

%\begin{figure}
%\captionsetup{labelformat=empty}
%\caption{
%\label{fig1}}
%\end{figure}
\begin{figure}[ht]
\centering
%\includegraphics[width=\textwidth]{Figure_01_HR.pdf}
\includegraphics[width=\textwidth]{grain_color.pdf}
\caption{Model geometry and field orientations. The most common morphology for authigenic greigite is truncated octahedral. To avoid the high density of field orientations necessary near the sphere poles when using a regular grid, 500 random field orientations from a distribution uniform over a sector of the sphere were chosen. The periodicity of the magnetocrystalline anisotropy means we only need to model the effects of field orientations on a sector of the sphere.}
\label{FIG_01}
\end{figure}


%\begin{figure}
%\captionsetup{labelformat=empty}
%\caption{
%\label{fig2}}
%\end{figure}
\begin{figure}[ht]
\centering
%\includegraphics[width=\textwidth]{Figure_02_HR.pdf}
\includegraphics[width=\textwidth]{comparison_rb.pdf}
\caption{Comparison between dipole and micromagnetic model. a) Dipole model; the FORC diagram (SF=4) of a noninteracting ensemble of SD greigite obtained using the model of \citet{ValdezGrijalva2017}. b) Micromagnetic model; the FORC diagram (SF=4) of a noninteracting ensemble of 30$\nm$-sized truncated octahedral greigite particles (this study). Up to 48$\nm$ the FORC diagram is that of an ensemble of coherent-rotating SD particles. For particles larger than 48$\nm$, magnetic vortex effects become noticeable.}
\label{FIG_02}
\end{figure}

%\begin{figure}
%\captionsetup{labelformat=empty}
%\caption{
%\label{fig3}}
%\end{figure}
\begin{figure}[ht]
\centering
%\includegraphics[width=\textwidth]{Figure_03_HR.pdf}
\includegraphics[width=\textwidth]{forcs_50-60.pdf}
\caption{FORC diagrams 50$\nm$ to 60$\nm$. All diagrams with SF=4. a) 50$\nm$; b) 52$\nm$; c) 54$\nm$; d) 56$\nm$; e) 58$\nm$; f) 60$\nm$. At these sizes, an ever larger fraction of the particles (those with their easy axes the furthest from the applied field) begin to rotate via nonuniform magnetisations, i.e., vortex nucleation.}
\label{FIG_03}
\end{figure}

%\begin{figure}
%\captionsetup{labelformat=empty}
%\caption{
%\label{fig7}}
%\end{figure}
\begin{figure}[ht]
\centering
%\includegraphics[width=\textwidth]{Figure_07_HR.pdf}
\includegraphics[width=\textwidth]{dayplot_MrsBc.pdf}
\caption{Day plot and remanence/coercivity against size. a) The Day plot shows the particles up to 60$\nm$ to be SD; however, we know from the micromagnetic solutions that vortices are formed from 50$\nm$ onward. Particles sized 62$\nm$ to 72$\nm$ plot in an unexpected region. Particles larger than 74$\nm$ plot as MD grains. b) Remanence (circles) and coercivity (triangles) against particle size. Remanences are high up to 72$\nm$ since vortices are only nucleated for negative values of the applied field. For particles larger than this the remanent state is a vortex and thus the sharp decrease in remanence. The coercivities decrease as soon as PSD behaviour is present. However, an anomalous region with increasing coercivities with size coincides with the sizes for which the Day plot is too, anomalous.}
\label{FIG_04}
\end{figure}

%\begin{figure}
%\captionsetup{labelformat=empty}
%\caption{
%\label{fig4}}
%\end{figure}
\begin{figure}[ht]
\centering
%\includegraphics[width=\textwidth]{Figure_04_HR.pdf}
\includegraphics[width=\textwidth]{forcs_62-72.pdf}
\caption{FORC diagrams 62$\nm$ to 72$\nm$. All diagrams with SF=4. a) 62$\nm$; b) 64$\nm$; c) 66$\nm$; d) 68$\nm$; e) 70$\nm$; f) 72$\nm$. These sizes correspond to the anomalous behaviour of the coercivities and the Day plot. At these sizes, a mixture of SD behaviour and SV behaviour is present, making them the most complex of all the diagrams calculated in this study.}
\label{FIG_05}
\end{figure}

%\begin{figure}
%\captionsetup{labelformat=empty}
%\caption{
%\label{fig5}}
%\end{figure}
\begin{figure}[ht]
\centering
%\includegraphics[width=\textwidth]{Figure_05_HR.pdf}
\includegraphics[width=\textwidth]{forcs_74-80.pdf}
\caption{FORC diagrams 74$\nm$ to 80$\nm$. All diagrams with SF=4. a) 74$\nm$; b) 76$\nm$; c) 78$\nm$; d) 80$\nm$. At these sizes (except 74$\nm$), all particles are in a remanent vortex state. The FORC diagrams are somewhat less complex than those for particles 62$\nm$ to 72$\nm$ as all traces of SD behaviour are gone. These sizes plot in the MD region of the Day plot. The maximum of the FORC distribution has moved to $B_c=\roughly$52$\mT$ along the $B_u=0$ axis, showing a disconnect with the coercivity of the ensembles which are lower than they would be for SD ensembles.}
\label{FIG_06}
\end{figure}

%\begin{figure}
%\captionsetup{labelformat=empty}
%\caption{
%\label{fig6}}
%\end{figure}
\begin{figure}[ht]
\centering
%\includegraphics[width=\textwidth]{Figure_06_HR.pdf}
\includegraphics[width=\textwidth]{FORC_avg_SF4_zoom_BcBu_flat.pdf}
\caption{FORC diagram averaged over the vortex-dominated ensembles. A flat particle size-distribution between 60$\nm$ and 80$\nm$ was used. Significant spread towards positive $B_u$ is observed as well as negative sources not present for SD ensembles. The spread towards $B_u>0$ has a distinctive tilted appearance, different from the vertical spread usually attributed to effects of interparticle magnetostatic interactions.}
\label{FIG_07}
\end{figure}
%-----------------------------------------------------

\end{document}
