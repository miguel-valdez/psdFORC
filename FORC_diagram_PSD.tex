\documentclass[review,authoryear]{elsarticle}
%-----------------------------------------------------

\usepackage{natbib}
\usepackage{amsmath}
\usepackage{graphicx}
\usepackage{url}
\usepackage{lineno}
\usepackage{caption}

\journal{Earth and Planetary Science Letters}

\bibliographystyle{elsarticle-harv}
\setcitestyle{authoryear}

\graphicspath{{./figs/}}

\newcommand{\rof}{{\raise.17ex\hbox{$\scriptstyle\sim$}}}
\newcommand{\nm}{\,\text{nm}}
\newcommand{\mT}{\,\text{mT}}
%-----------------------------------------------------

\begin{document}
\allowdisplaybreaks

\begin{frontmatter}

\title{Magnetic vortex effects on the first order reversal curve (FORC) diagram of greigite dispersions}

\author[ic]{Miguel~A.~Valdez-Grijalva\corref{co}}
\ead{m.valdez-grijalva13@imperial.ac.uk}
\cortext[co]{Corresponding author}

\author[ic]{Adrian~R.~Muxworthy}
\author[ed]{Wyn~Williams}
\author[ed]{Lesleis~Nagy}
\author[ed]{P{\'a}draig~{\'O}~Conbhu{\'i}}
\author[au]{Andrew~P.~Roberts}
\author[au]{David~Heslop}

\address[ic]{Department of Earth Science and Engineering, Imperial College London, SW7 2BP, UK}
\address[ed]{School of GeoSciences, University of Edinburgh, EH9 3FE, UK}
\address[au]{Research School of Earth Sciences, Australian National University, 2601, Australia}

\begin{abstract}
In this study, the properties of non-interacting, randomly oriented dispersions of greigite (Fe$_3$S$_4$) in the single-domain (SD) to single-vortex (SV) size range are investigated via micromagnetic calculations. The FORC diagrams were found to be highly sensitive to the magnetic domain state of the particles. The SD signal ($<50\nm$) was found to be in excellent agreement with previous SD coherent-rotation studies. A transitional range from $\rof$50$\nm$ to $\rof$70$\nm$ was identified for which a mixture of SD and SV behaviour during hysteresis produces complex FORC diagrams. Particles $> \rof 76\nm$ show a purely SV behaviour with the remanent state for all particles in the ensemble a vortex state. It is concluded that for SV ensembles the FORC distribution peak should not be interpreted as the coercivity of the sample but as a vortex annihilation field on the path to saturation.
\end{abstract}

\begin{keyword}
Greigite \sep Single Vortex \sep \sep Micromagnetic \sep FORC diagram
\PACS 91.25.F- \sep 91.60.Pn \sep 91.25.fa
\end{keyword}

\end{frontmatter}

\linenumbers
%-----------------

\section{Introduction}

First-order reversal curve (FORC) diagrams are a powerful tool in rock magnetic studies which allow for mineral and domain state identification as well as quantification of magnetostatic interactions between particles \citep{Pike1999,Roberts2000,Roberts2014,Dumas2007}. As such, they have been the subject of numerical studies aiming to relate the individual behaviour of the magnetic particles and small assemblages to the experimental bulk properties \citep{Pike1999,Carvallo2003,Carvallo2006,Muxworthy2004,Muxworthy2005,Newell2005,Harrison2014,ValdezGrijalva2017}.\par
%The FORC diagram properties of single-domain (SD) particles have been studied in both the noninteracting \citep{Newell2005} and interacting cases \citep{Muxworthy2004,Muxworthy2005}. However, iron-rich minerals of the utmost importance to the Earth Sciences like magnetite (Fe$_3$O$_4$) and greigite (Fe$_3$S$_4$) do not posses a uniaxial MCA but rather, cubic. Also, these studies do not take into account size effects as they effectively treat each particle as a single magnetic dipole.\par
%The noninteracting FORC signal of SD particles with cubic magnetocrystalline anisotropy (MCA) was recently calculated by \citet{ValdezGrijalva2017} using a simple dipole model. Whereas, \citet{Carvallo2003,Carvallo2006,Muxworthy2004,Muxworthy2005} employed numerical micromagnetic models to study the effects of magnetostatic interactions in small assemblages of SD magnetite. However, the FORC diagram of particles with nonuniform magnetisations, specially those in a single-vortex (SV) state, is not yet fully understood.\par

With the exception of \citet{Carvallo2003}, all of these previous numerical studies have concentrated on the FORC diagrams of single-domain (SD) particles, usually of magnetite. They have shown that uniaxial SD particles have patterns \citep{Muxworthy2004,Newell2005}, that are distinct from SD materials with cubic anisotropy \citep{Muxworthy2004,ValdezGrijalva2017}. However, it is well-documented that most natural systems have magnetic signals dominated by larger grains with more complex magnetic domain states, e.g., pseudo-single domain \citep{Roberts2017}. Grains just above the SD threshold ($\rof 80 \nm$ for magnetite, $\rof 54 \nm$ for greigite), are typically in a single-vortex (SV) state. The SV state dominates magnetic structures over an order of magnitude \citep{Nagy2017, ValdezGrijalva2017B}, a range much wider than the stable SD range. It has also been shown that SV states are geologically meta-stable and retain relatively high remanences \citep{Nagy2017, ValdezGrijalva2017B}.\par

Previous experimental studies on nano-patterned arrays with particles in the SV state \citep{Pike1999B,Dumas2007} have shown that the FORC diagram patterns exhibited by SV-dominated systems are significatively more complex than the SD signal. \citet{Pike1999B} focussed on obtaining vortex nucleation/annihilation fields in a patterned array of sub-micron Co dots. \citet{Dumas2007} emphasised the sensitivity of the FORC method to effects of particle size in size-controlled sub-100$\nm$ Fe dot arrays.\par

Studies of 2D nano-patterned arrays \citep{Pike1999B,Dumas2007} are difficult to relate to the behaviour of natural systems. In natural samples, particles with varying size and orientations are dispersed in space. Thus, it is important to understand the contribution of dispersions of randomly aligned SV particles to FORC diagrams. Numerical modelling can aid the study of such systems. \citet{Carvallo2003} used a finite-difference model to calculate the FORC distributions of SV magnetite particles, however, their model was too low resolution to now be considered reliable, did not study random distributions nor did it model realistic grain morphologies.\par

In this study, we employ a micromagnetic finite element method (FEM) to study the FORC diagram properties of noninteracting ensembles of greigite (Fe$_3$S$_4$); the iron-sulphide counterpart to magnetite. Recent interest in greigite comes from both its promising properties for material scientists \citep{Guowei2014} and the abundance of this mineral in sedimentary rocks for Earth scientists \citep{Roberts2011}. The unstructured discretisation of FEMs allows us to study realistic shapes of greigite particles observed in nature. We obtain FORC diagrams of noninteracting dispersions of greigite with sizes 30--80$\nm$; this size range covers the SD--SV threshold \citep{ValdezGrijalva2017B}. We determine the onset of SV behaviour and its consequences for FORC diagram interpretation.\par
%-----------------------------------------------------

\section{Methodology}
\subsection{The micromagnetic algorithm}
A ferromagnetic material{\textemdash}neglecting thermal and magnetostrictive effects{\textemdash}has a Gibbs free-energy functional given by \citep{Brown}
{\par\nobreak\noindent}
\begin{equation}
E_\text{G} = \int_{\Omega} (\phi_{\text{exchange}} + \phi_{\text{anisotropy}} + \phi_{\text{stray}} + \phi_{\text{external}})\,\text{d}^3 \boldsymbol{r},
\end{equation}
with $\Omega$ the ferromagnetic volume. Here,
{\par\nobreak\noindent}
\begin{equation}
\phi_{\text{exchange}}=A|\nabla\boldsymbol{m}|^2
\end{equation}
 with the reduced magnetisation vector $\boldsymbol{m}$ and the exchange stiffness constant $A$, is the expression for the energy density due to the quantum-mechanical exchange forces \citep{Landau1935}.
{\par\nobreak\noindent}
\begin{equation}
\phi_{\text{anisotropy}}=\frac{K_1}{2}\sum_{i\neq j}\gamma_i^2\gamma_j^2 + K_2\prod_i\gamma_i^2,
\end{equation}
with $\gamma_i$ the direction cosines and $K_1, K_2$ the first and second magnetocrystalline anisotropy (MCA) constants, is the MCA energy density in the cubic anisotropy system. In terms of the reduced magnetisation vector components this becomes:
{\par\nobreak\noindent}
\begin{equation}
\phi_{\text{anisotropy}}=K_1(m_x^2m_y^2+m_y^2m_z^2+m_z^2m_x^2),
\end{equation}
where $K_2$ has been neglected since $K_1$ is the dominant term at room temperature. The magnetostatic self-energy density is given by
{\par\nobreak\noindent}
\begin{equation}
\phi_{\text{stray}}=-\frac{\mu_0M_\text{S}}{2}\boldsymbol{m}\cdot\boldsymbol{H}_{\text{stray}},
\end{equation}
with $\boldsymbol{H}_{\text{stray}}$ the stray field produced by the ferromagnetic body and $M_\text{S}$ the saturation magnetisation. Finally, the energy density due to an external magnetic field $\boldsymbol{H}_{\text{external}}$ is
{\par\nobreak\noindent}
\begin{equation}
\phi_{\text{external}}=-\mu_0M_\text{S}\boldsymbol{m}\cdot\boldsymbol{H}_{\text{external}}.
\end{equation}\par

It is known that the system will be spontaneously driven towards an equilibrium state with a locally minimal magnetic Gibbs free-energy \citep{Brown}. In this study we utilise a modified gradient descent method with the aim of finding the equilibrium magnetisation \citep{OConbhui2017}.\par

The discretisation of the spatial domain is achieved by a decomposition of the volume into tetrahedral elements. This allows for arbitrary geometries to be modelled. To model accurately nonuniform magnetisations the spatial discretisation in the model should be smaller than the exchange length $l_\text{exch.} = \sqrt{2A/\mu_0M_\text{S}^2}$ \citep{Rave1998}, which for greigite is $l_\text{exch.} \approx 6.6\, \text{nm}$; a maximum element size of 5$\nm$ has been chosen for all the models. The non-local problem of calculating the stray field is resolved by a hybrid finite-element/boundary-element formulation \citep{Fredkin1990}.\par

The magnetic parameters of greigite used in this investigation are: the saturation magnetisation $M_\text{S}=3.51\,\mu_\text{B}\,\text{p.c.u.}$ or $\rof 2.7 \times 10^5\,\text{A/m}$ \citep{Guowei2014}; the exchange stiffness constant $A=2\times10^{-12}\,\text{J}/\text{m}$ \citep{Chang2008}; and the first MCA constant $K_1=-1.7\times10^4\,\text{J}/\text{m}^3$ \citep{Winklhofer2014}. This set of parameters has been used in recent numerical studies of greigite \citep{ValdezGrijalva2017B,ValdezGrijalva2017}.\par

\subsection{The FORC model}
FORC diagrams are constructed from a class of partial hysteresis curves called first order reversal curves, each starting at some value of the applied field $B_a$ along the main hysteresis branch. A magnetisation function on two variables $M=M(H_a, H_b)$ is thus obtained. The FORC distribution $\rho$ is then defined as \citep{Roberts2000}:
{\par\nobreak\noindent}
\begin{equation}
\rho=-\frac{1}{2}\frac{\partial^2 M}{\partial H_a \partial H_b}.
\end{equation}\par

Once $M(H_a, H_b)$ is obtained, the calculation of $\rho(H_a, H_b)$ is done by least-square fitting a degree 2 polynomial surface $a_0 + a_1 H_a + a_2 H_b + a_3 H_a H_b + a_4 H_a^2 + a_5 H_b^2 = 0$ on a subgrid of $M(H_a, H_b)$ centered around $H_a, H_b$ determined by the so-called smoothing factor (SF) and including (2$\times$SF + 1)$^2$ points; the value of $\rho$ is then simply $-a_3/2$ \citep{Pike1999}. Contour plots of the FORC distribution are called FORC diagrams. Usually they are presented in the rotated axes $H_c=(H_b - H_a)/2$, $H_u=(H_b + H_a)/2$.\par

Random field orientations from a distribution uniform over a sector of the sphere have been chosen (Fig. \ref{FIG_01}). We have chosen to use 500 field orientations as a good compromise between accuracy and calculation speed (Fig. \ref{FIG_02}). Also, we note that for each particle/field-orientation the hysteresis curve consists mostly of reversible motion of the magnetisation; thus, we only need to calculate the main branch of the hysteresis loop and the few reversal curves starting at the different switching fields along the main branch \citep{ValdezGrijalva2017}. These simplifications vastly reduce the amount of calculations needed without loss of generality. The external-field rate of change for all models was 1$\mT$ with a saturation field of 250$\mT$ thus, 501 reversal curves were calculated for each particle/field-orientation.\par

Scanning electron microscopy and transmission electron microscopy micrographs of naturally occurring greigite samples \citep{Snowball1997,Vasiliev2008,Roberts2015} reveal that greigite tends to grow authigenically as well-defined regular truncated octahedral particles. Micromagnetic calculations for truncated octahedral greigite particles show the SD--SV threshold to be $\rof 54\nm$ \citep{ValdezGrijalva2017B}. In this study we model the FORC diagram of noninteracting ensembles of greigite particles sized 30--80$\nm$ (size normalised to the volume of a cube) every 2$\nm$ as this range covers the transition from SD to SV behaviour.\par

%-----------------------------------------------------

\section{Results}
For the ensembles with particles $<50\nm$ the hysteretic behaviour is dominated by pure SD particles with coherent rotations (Fig. \ref{FIG_02}). This is evidenced by the FORC diagrams of these ensembles showing the same, distinct pattern exhibited by pure SD, coherent-rotating greigite particles \citep{ValdezGrijalva2017}. The diagrams for the particles $<50\nm$ here obtained with the micromagnetic algorithm (Fig. \ref{FIG_02}b) are offset $\rof 3\mT$ towards the left compared to the dipolar model \citep{ValdezGrijalva2017} (Fig. \ref{FIG_02}a); lower coercivities due to the micromagnetic algorithm including magnetostatic self-interaction effects as well as flowering (small deviations from a perfect SD structure) account for this effect.\par

Particles with cubic anisotropy exhibit hysteresis behaviour departing from the simple hysteron with one plus-state and one minus-state. There exist intermediate states along their hysteresis curves even when they are SD \citep{ValdezGrijalva2017}, especially those with easy axes alignment far from the applied field orientation. The tilted, elongated, negative-valued ridge (Fig. \ref{FIG_02}) is a consequence of the cubic anisotropy and is produced by the fraction of particles with hard-axes closely aligned with the applied field. These particles have the lowest switching fields: from the plus-state to an intermediate state at $B=\mu_0 H = B^{+}_{*}$ and from the intermediate state to the minus-state at $B=B^{*}_{-}$. Reversal curves with $B^{*}_{-} < B_a = \mu_0 H_a < B^{+}_{*}$ experience a sharp upward discontinuity at $B_b = \mu_0 H_b = B^{*}_{+} < |B^{+}_{*}|$ when the hard-aligned particles return to the plus-state from their intermediate states. The combination of this type of irreversible events in the hard-aligned particles cause the local peak at $B_c=\rof 15 \mT, B_u=\rof -3 \mT$ (Fig. \ref{FIG_02}b). On the reversal curves with $B_a < B^{*}_{-}$ the hard-aligned particles are initially in the minus-state and undergo irreversible rotation to an intermediate state on the path to positive saturation at $B=B^{-}_{*} = |B^{+}_{*}|$ due to the symmetry of the particles and the lack of magnetostatic interactions. The combination of these irreversible events causes a negative FORC distribution source at $B_a=B^{*}_{-}, B_b=B^{*}_{+}$. The sum effect of this type of sources for many particles with a distribution of the aforementioned switching fields produces the highly structured, elongated negative contribution observed in all SD ensembles.\par

The fraction of particles with easy axis alignment close to the applied field orientation exhibit hysteron-like behaviour, i.e., just two switching fields: from the plus-state to the minus-state $B^{+}_{-}$ and viceversa $B^{-}_{+}$. The lack of interactions and symmetry of the particles ensure that $|B^{+}_{-}| = B^{+}_{-}$. Thus, this fraction of particles produce FORC distribution sources at $B_a=B^{+}_{-}, B_b=B^{-}_{+}$. These type of sources, then, accumulate on the line $B_a=-B_b$. This type of irreversible events account for the most drastic changes in the magnetisation of the ensemble and thus account for the high slopes around the coercive field of the sample. This makes the position of the FORC diagram peak coincide with the coercivity $B_\text{C}=\mu_0 H_\text{C}=\rof 24\mT$ of the SD ensembles.\par

For particles with size $d> 50\nm$ the FORC diagrams depart from coherent rotation SD-like behaviour as the tight boomerang-shaped FORC diagram pattern exhibited by the SD greigite becomes more fragmented (Figs. \ref{FIG_03}). This change is initially driven by the particles with hard axes close to the applied field nucleating hard-aligned vortices (HAVs) \citep{ValdezGrijalva2017B} as intermediate meta-stable states during hysteresis. Even though the nucleation of HAVs is happening for particles with sizes below the zero-field SD--SV threshold $d_0=\rof$54$\nm$ \citep{ValdezGrijalva2017B}, this is expected as the nucleation of a vortex greatly reduces the Gibbs free-energy due to the external and demagnetising (stray) fields. A corollary of this is that a fraction of particles (those with easy axis alignment close to the applied field) above the zero-field SD--SV threshold can remain in a SD state during the entirety of hysteresis. These effects are due to the distortion of the zero-field energy landscape by the applied field.\par

An appreciable positive source in the FORC distribution appears along the $B_u=\mu_0 H_u=0$ axis at $B_c=\mu_0 H_c=\rof 52\mT$ for the ensembles with particles $\geq 50\nm$ (Fig. \ref{FIG_03}); this contribution represents the unravelling of vortex states on the way back to positive saturation. The elongated, negative ridge of the SD diagram moves towards lower values of $B_c, B_u$ and the first sources for $B_u > 0$ begin to form; these are elongated features at a 45 degree angle from the $B_u=0$ axis, different from the vertical widening usually attributed to the presence of magnetostatic interactions \citep{Muxworthy2004,Muxworthy2005}.\par

For the particles slighly below and above the SD--SV threshold $d_0$, vortex nucleation occurs only for negative values of the applied field, thus the noticeable changes in the FORC diagram (Fig. \ref{FIG_03}) are not reflected in changes in the remanence of saturation $M_\text{RS}$ (the value of the magnetisation when the applied field vanishes during hysteresis) to saturation magnetisation $M_\text{S}$ ratio up to 72$\nm$, whereas the coercivities sharply decrease above 48$\nm$ (Fig. \ref{FIG_04}b). The monotonically-decreasing trend of the coercivites is preserved up to 62$\nm$ when the coercivity rises from $B_{\text{C}}=\rof 15 \mT$ to $B_{\text{C}}=\rof 20 \mT$ (coercivity $B_{\text{C}}$ not to be confused with the FORC diagram axis $B_c$) for $d=68\nm$.\par

On increasing size, the coercivities further decrease accompanied by a sharp decrease in the remanence (Fig. \ref{FIG_04}b). The drop in the remanence values is driven by particles nucleating vortices at $B_a>0$ for $d \geq 68\nm$. For $d \geq 76\nm$, all particles nucleate vortices which become the remanent magnetic domain state; this is reflected in the Day plot \citep{Day1977}, a scatter plot of the $M_\text{RS}/M_\text{S}$ ratio against the coercivity of remanence $B_\text{CR}$ (the field necessary so the magnetisation vanishes when the field goes to zero) to $B_\text{C}$ ratio by the particles sized 76$\nm$ and larger plotting in the region usually designated as pseudo single-domain (PSD) (Fig. \ref{FIG_04}a), synonymous with SV in this paper.\par

Particles sized 62$\nm$ to 72$\nm$ move away from the Day plot region usually attributed to SD grains (Fig. \ref{FIG_04}a) to a region with high remanence but larger values of $B_{\text{CR}}/B_{\text{C}}$ ratio. These sizes coincide with the anomalous behaviour of the coercivity rising for these sizes (Fig. \ref{FIG_04}b). The FORC diagrams for these sizes are the most complex of all, showing a variety of X-shaped features (Fig. \ref{FIG_03}b,c) and the appearance of a substantial symmetric negative source at $B_c=\rof$5$\mT$, $B_u=\rof$-10$\mT$ (Fig. \ref{FIG_03}a--c). The elongated, negative ridge becomes more faint with size. Whereas, the positive sources for $B_u>0$ become stronger and moving towards the $B_c=0$ axis with size. Strong, positive FORC distribution sources for $B_u>0$ along the $B_c=0$ axis are expected for the larger multi-domain (MD) grains \citep{Roberts2006}, and this behaviour might be representative of that tendency.\par

The elongated, negative ridge typical of cubic MCA SD particles \citep{ValdezGrijalva2017} disappears for the particles $\geq 76 \nm$ (Fig. \ref{FIG_03}d). The symmetric negative feature near the origin becomes stronger and of a magnitude comparable to the largest positive source. For $d=76,78\nm$ the negative source is larger (absolute value) than the distribution peak (Fig. \ref{FIG_03}d). Towards 80$\nm$ a faint negative source appears close to the positive source situated along the $B_u=0$ axis (Fig. \ref{FIG_05}). These diagrams represent the contribution of purely SV particles, that is, ensembles of particles that are all in a vortex remanent state. It is logical that these are somewhat less complex than the diagrams of ensembles comprised of a fraction of particles still in the SD state as well as SV.\par

Closer examination of the SV FORC signal and the micromagnetic solutions reveal what the different FORC distribution sources mean and how they relate to the raw hysteresis data (Fig. \ref{FIG_05}). On the hysteresis main branch, particles with hard axes closely aligned with the applied field nucleate hard-aligned vortices at high values of the applied field (Fig. \ref{FIG_06}); as the field is decreased below $\rof 12\mT$ these vortices irreversibly rotate to an easy axis alignment. As the field is increased on the reversal curves with $\rof 0\mT < B_a < \rof 12\mT$ these vortices irreversibly switch back to a hard alignment at $B_b=\rof 28 \mT$ creating a local peak at $B_c = \rof 16 \mT,\, B_u = \rof 12 \mT$ (Fig. \ref{FIG_05}, region 1); this is manifested in the raw hysteresis data by the smoothed discontinuity at $B=\rof 28 \mT$ whereas the reversible motion traced by the reversal curves around this region accounts for the tilted, elongated source surrounding the local peak.\par

During the hysteresis, as the remanent state is approached, all particles have nucleated vortices: particles with easy axes alignment close to the applied field directly nucleate an easy aligned vortex while the rest nucleate vortices initially oriented along hard $<100>$ or $<110>$ directions for higher values of the applied field which irreversibly rotate to an easy axis alignment as the field approaches zero. The latter fraction of particles are then affected most by the torques caused by the applied field and between $\rof -10 \mT$ to $\rof -20 \mT$ undergo irreversible rotations to intermediate positions, i.e., hard alignments at an angle not opposite to the one they nucleated from. Then, on the reversal curves with $\rof -20\mT < B_a < \rof -10\mT$ these vortices rotate back to the initial easy axis alignment at $B_b=\rof 4 \mT$. The combination of these irreversible events creates the lowest, negative FORC distribution source (Fig. \ref{FIG_05}, region 2). Further increase of the applied field to $\rof 30 \mT$ causes these vortices to switch to the initial hard position they nucleated from. These events cause the tilted, elongated source by the FORC distribution minimum (Fig. \ref{FIG_05}, region 3).\par

As the applied field approaches $\rof -52 \mT$ during hysteresis, the vortices of particles with easy axis alignment close to the applied field annihilate (Fig. \ref{FIG_06}). The reversal curves with $\rof -80\mT < B_a < \rof -52\mT$ trace lower slopes with decreasing $B_a$ due to the combined reversible motion of vortices and single-domains. This is the source of the faint negative-valued contribution for $B_u < \rof 45 \mT$ (Fig. \ref{FIG_05}, region 4). On increasing $B_b$ on these curves, nucleation of easy aligned vortices occurs at $\rof -5 \mT$ creating the source by the aforementioned negative source (Fig. \ref{FIG_05}, region 5); this corresponds with the smoothed discontinuity on the hysteresis curves as the field approaches zero from the left. Increasing the applied field towards positive values causes the easy aligned vortices of particles with hard axes close to the applied field to switch to hard alignments at $\rof 28 \mT$, this creates an appreciable negative-valued FORC distribution source by the distribution peak (Fig. \ref{FIG_05}, region 6). The distribution peak (Fig. \ref{FIG_05}, region 7) corresponds to the average annihilation field of the vortices on the reversal paths to positive saturation.\par

There is a large spread in the vortex nucleation and annihilation fields on the hysteresis curve (Fig. \ref{FIG_06}). Particles with hard axes alignment close to the applied field nucleate hard-aligned vorticesfor fields as high as $\rof 200 \mT$ and annihilate on the opposite side of the particle for equally high values (absolute value). However, these nucleation and annihilation events have a negligible contribution to the FORC diagram as the change in magnetisation of a particle nucleating/annihilating a hard-aligned vortex from/to a SD state can be as low as $1 \%$.\par

%-----------------------------------------------------
\section{Discussion}
Comparison with the coherent-rotating dipole model of \citet{ValdezGrijalva2017} shows excellent agreement (Fig. \ref{FIG_02}). This confirms the accuracy of our model using a number of random field orientations instead of a larger number of field orientations on a regular grid.\par

The FORC diagram of the SD coherent-rotating particles shows the same general features as those obtained for weakly interacting SD particles with cubic MCA by \citet{Harrison2014}, i.e., a positive-valued ridge along the $H_c$ axis, slightly offset towards $H_u<0$ values and a tilted, negative-valued ridge on the lower half of the FORC diagram space. For this type of ensembles the horizontal spread along the $H_c$ axis roughly corresponds with the density of switching fields of the differently oriented particles and the FORC distribution peak position is. The negative ridge is indicative of intermediate states along the hysteresis curve and therefore, of cubic anisotropy SD particles \citep{ValdezGrijalva2017}. This elongated negative contribution is potentially a FORC signal unique to noninteracting to weakly interacting SD particles with cubic MCA.\par

Whereas the pure SD signal produces a tight, boomerang-shaped pattern, the increasing with size effects of nonuniform SV magnetic structures is to fragment this pattern. The FORC distribution peak is moved towards higher $B_c$ values along the $B_u=0$ axis. Paradoxically, as this occurs, the coercivity of the ensembles decrease. This paradox has been previously observed in synthetic size-controlled samples of sub-100$\nm$ Fe dots \citep{Dumas2007}.\par

Fragmentation of the FORC diagram for non-uniformly magnetised particles was previously observed in experimental studies by \citep{Pike1999B,Dumas2007} and in numerical models by \citet{Carvallo2003}; however these studies included (at least weak) magnetostatic interactions between the particles and grain elongations. The trend is nevertheless clear and is representative of the complex self-interactions brought about by the nonuniform structures and the multiple vortex nucleation/annihilation fields \citep{Pike1999B}. It is difficult to compare our results to the FORC signals measured by \citet{Muxworthy2006,Krasa2011} as they studied large, synthetic magnetite particles in the multi-domain state.\par

In cubic anisotropy particles, differently aligned vortices have different average magnetisations, energies \citep{ValdezGrijalva2017}. \citet{Pike1999B} obtained nucleation and annihilation fields of magnetic vortices in nano-patterned Co dots. Our results agree with their finding that nucleation and annihilation fields are usually asymmetric. However, \citet{Pike1999B} studied elongated disc-like particles where the vortex cores are always perpendicular to the particle plane and mostly perform reversible motion from nucleation to annihilation as they traverse the particle. In this study we have demonstrated that the different sources on the FORC diagram, for SV ensembles, are related to a variety of vortex nucleation and annihilation fields depending on particle alignment and the presence of distinctly different vortex states. We found that nucleation/annihilation of hard-aligned vortices has little effect on the FORC diagram at the onset of nucleation/annihilation. Rather, the presence of these vortices is revealed as they irreversibly rotate to easy aligned or other directions.\par

Averaged-over-size FORC diagrams were obtained for flat particle size distributions between 30$\nm$ and 80$\nm$ (Fig. \ref{FIG_07}a) and between 60$\nm$ and 80$\nm$ (Fig. \ref{FIG_07}b). The diagram for the size distribution including SD particles (Fig. \ref{FIG_07}a) shows the boomerang-shaped pattern surrounded by a variety of other more complez sources. This pattern is strikingly similar to the butterfly-shaped patterns observed by \citet{Dumas2007} for their samples including both SD and SV particles. The FORC distribution peak position coincides with the ensemble coercivity while still showing a somewhat strong source corresponding to the annihilation field of easy aligned vortices.\par

These diagrams have a significative spread towards positive $B_u$. This effect is purely due to domain state, not magnetostatic interactions. The maximum for the SV-dominant diagram (Fig. \ref{FIG_07}b) occurs along the $B_u=0$ axis at $B_c=\rof$52$\mT$ and thus not related to the coercivity of the ensemble ($B_\text{C}=\rof$16$\mT$). This was observed by \citet{Dumas2007}. This is a departure from the usual interpretation of FORC diagram, i.e., that the FORC diagram maps the coercivities. This interpretation holds for the SD coherent-rotation grains, as this source coincides with the value of the ensemble coercivity. It does not hold, however, for the SV grains as their coercivity decreases with size while the position of the maximum moves towards higher values of $B_c$. Instead, for SV grains this peak, and indeed most of the FORC diagram sources, should be interpreted as vortex nucleation/annihilation fields.\par
%-----------------------------------------------------

\section{Conclusion}
A micromagnetic FEM was employed for the calculation of the FORC diagrams of noninteracting ensembles of greigite across a size range that spans the SD to SV threshold. 500 random orientations from a distribution uniform over a sector of the sphere were used for each size. This choice was found to be in excellent agreement with previous calculations \citep{ValdezGrijalva2017} for coherent-rotating SD particles using a regular grid of field orientations.\par

The FORC diagrams showed to be extremely sensitive to the domain state of the particles. When even a small fraction of particles starts to nucleate vortices, at $d=\rof$50$\nm$, this is reflected in the FORC diagram. The same cannot be said of the Day plot which `treats' the particles up to 60$\nm$ as strictly SD. Anomalous behaviour for particles sized 62$\nm$ to 72$\nm$ consisting in an increase of the coercivity with size was found; these particles plot in an unexpected region of the Day plot. The anomaly disappears for particles $>72\nm$, and when $d>76\nm$ they appear in the region usually attributed to PSD grains.\par

Detailed analysis of the FORC diagram and micromagnetic solutions of $d=80\nm$ particles reveals the meaning of the FORC diagram for SV ensembles as a map of vortex nucleation/annihilation fields. Previous notions on how the FORC diagrams should be interpreted as a coercivity map need to be updated. This is illustrated by the disconnect on what the FORC distribution peak mean for SD or SV particles. For SD particles, the typical interpretation of the peak position coinciding with the coercivity of the sample holds. However, for SV-dominated samples, the position of the maximum occurs at a value much higher than the coercivity of the sample and is actually related to the annihilation field of easy aligned vortices on the path to positive saturation.\par
%-----------------------------------------------------

\section*{Acknowledgments}
This research was partially funded by Instituto Mexicano del Petr\'oleo (M. A. Valdez-Grijalva) as well as by NERC grant NE/J020508/1 (A. R. Muxworthy and W. Williams).\par
%-----------------------------------------------------

\section*{References}
\def\urlprefix{}
\bibliography{references}
\newpage
%-----------------------------------------------------

%\begin{figure}
%\captionsetup{labelformat=empty}
%\caption{
%\label{fig1}}
%\end{figure}
\begin{figure}[ht]
\centering
%\includegraphics[width=\textwidth]{Figure_01_HR.pdf}
\includegraphics[width=\textwidth]{grain_color.pdf}
\caption{Model geometry and field orientations. The most common morphology for authigenic greigite is truncated octahedral. To avoid the high density of field orientations necessary near the sphere poles when using a regular grid, 500 random field orientations from a distribution uniform over a sector of the sphere were chosen. The periodicity of the magnetocrystalline anisotropy and particle symmetry allows to only model the effects of field orientations on a sector of the sphere without loss of generality.}
\label{FIG_01}
\end{figure}


%\begin{figure}
%\captionsetup{labelformat=empty}
%\caption{
%\label{fig2}}
%\end{figure}
\begin{figure}[ht]
\centering
%\includegraphics[width=\textwidth]{Figure_02_HR.pdf}
\includegraphics[width=\textwidth]{comparison.pdf}
\caption{Comparison between dipole and micromagnetic model. a) Dipole model; the FORC diagram (SF=4) of a noninteracting ensemble of SD greigite obtained using the model of \citet{ValdezGrijalva2017}. b) Micromagnetic model; the FORC diagram (SF=4) of a noninteracting ensemble of 30$\nm$-sized truncated octahedral greigite particles. Up to 48$\nm$ the FORC diagram is that of an ensemble of coherent-rotating SD particles. For particles larger than 48$\nm$, magnetic vortex effects become noticeable.}
\label{FIG_02}
\end{figure}

%\begin{figure}
%\captionsetup{labelformat=empty}
%\caption{
%\label{fig3}}
%\end{figure}
\begin{figure}[ht]
\centering
%\includegraphics[width=\textwidth]{Figure_03_HR.pdf}
\includegraphics[width=\textwidth]{FORCs_54_60_66_76.pdf}
\caption{FORC diagrams with increasing vortex effects. All diagrams with SF=4. a) 54$\nm$; b) 60$\nm$; c) 66$\nm$; d) 76$\nm$. At these sizes, an ever larger fraction of the particles begin to rotate via nonuniform magnetisations, i.e., vortex nucleation. At 76$\nm$ all particles are in the single vortex remanent state.}
\label{FIG_03}
\end{figure}

%\begin{figure}
%\captionsetup{labelformat=empty}
%\caption{
%\label{fig7}}
%\end{figure}
\begin{figure}[ht]
\centering
%\includegraphics[width=\textwidth]{Figure_07_HR.pdf}
\includegraphics[width=\textwidth]{dayplot_MrsBc.pdf}
\caption{Day plot and remanence/coercivity against size. a) The Day plot shows the particles up to 60$\nm$ to be SD; however, we know from the micromagnetic solutions that vortices are formed from 50$\nm$ onward. Particles sized 62$\nm$ to 72$\nm$ plot in an unexpected region. Particles larger than 74$\nm$ plot as MD grains. b) Remanence (circles) and coercivity (triangles) against particle size.}
\label{FIG_04}
\end{figure}

%\begin{figure}
%\captionsetup{labelformat=empty}
%\caption{
%\label{fig4}}
%\end{figure}
\begin{figure}[ht]
\centering
%\includegraphics[width=\textwidth]{Figure_04_HR.pdf}
\includegraphics[width=\textwidth]{forc_80nm_annotated.pdf}
\caption{FORC diagram (SF=4) and hysteresis curves of 80$\nm$-sized particles. Annotations link the FORC diagram sources to the raw hysteresis curves. See text for a detailed relation.}
\label{FIG_05}
\end{figure}

%\begin{figure}
%\captionsetup{labelformat=empty}
%\caption{
%\label{fig5}}
%\end{figure}
\begin{figure}[ht]
\centering
%\includegraphics[width=\textwidth]{Figure_05_HR.pdf}
\includegraphics[width=\textwidth]{nucleation_annihilation_fields.pdf}
\caption{Vortex nucleation/annihilation fields. a) Scatter plot of annihilation field against nucleation field. Three trends are observed depending on whether the nucleated/annihilated vortex has an easy, hard or other alignment. b) Vortex core angle with an easy direction against the nucleation/annihilation fields (circles, triangles respectively).}
\label{FIG_06}
\end{figure}

%\begin{figure}
%\captionsetup{labelformat=empty}
%\caption{
%\label{fig5}}
%\end{figure}
\begin{figure}[ht]
\centering
%\includegraphics[width=\textwidth]{Figure_05_HR.pdf}
\includegraphics[width=\textwidth]{size_averaged.pdf}
\caption{Averaged-over-size FORC diagrams (SF=4) and raw hysteresis curves. Flat size distributions used for particles $30\nm \leq d \leq 80\nm$ (a, c) and particles $60\nm \leq d \leq 80\nm$ (b, d).}
\label{FIG_07}
\end{figure}

%-----------------------------------------------------

\end{document}
