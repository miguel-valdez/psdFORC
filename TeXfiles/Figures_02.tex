\documentclass[review]{elsarticle}
%-----------------------------------------------------

\usepackage{amsmath}
\usepackage{graphicx}

\graphicspath{{./figs/}}

\newcommand{\roughly}{{\raise.17ex\hbox{$\scriptstyle\sim$}}}
\newcommand{\nm}{\,\text{nm}}
\newcommand{\mT}{\,\text{mT}}

%-----------------------------------------------------

\makeatletter
\renewcommand{\fnum@figure}{Figure 2}
\makeatother

\thispagestyle{empty}

\begin{document}
\allowdisplaybreaks

\begin{figure}[ht]
\centering
%\includegraphics[width=\textwidth]{FIG_02BW.pdf}
%\includegraphics[width=\textwidth]{FIG_02BW_LR.pdf}
\includegraphics[width=\textwidth]{FIG_02.pdf}
%\includegraphics[width=\textwidth]{FIG_02_LR.pdf}
\caption{Comparison between FORC diagrams produced with dipole and micromagnetic models. a) Dipole model; FORC diagram (SF=4) for a non-interacting ensemble of idealised (size-independent) SD greigite particles obtained using the model of Valdez-Grijalva and Muxworthy (2018). b) Micromagnetic model; FORC diagram (SF=4) for a non-interacting ensemble of 30$\nm$ truncated octahedral greigite particles. Up to 48$\nm$, the FORC diagram is that of an ensemble of coherently rotating SD moments. For particles larger than 48$\nm$, magnetic vortex effects become important. Dashed contour lines denote negative $\rho$ values.}
\label{FIG_02}
\end{figure}

\end{document}
