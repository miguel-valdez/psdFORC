\documentclass[review]{elsarticle}
%-----------------------------------------------------

\usepackage{amsmath}
\usepackage{graphicx}

\graphicspath{{./figs/}}

\newcommand{\roughly}{{\raise.17ex\hbox{$\scriptstyle\sim$}}}
\newcommand{\nm}{\,\text{nm}}
\newcommand{\mT}{\,\text{mT}}

%-----------------------------------------------------

\makeatletter
\renewcommand{\fnum@figure}{Figure 3}
\makeatother

\thispagestyle{empty}

\begin{document}
\allowdisplaybreaks

\begin{figure}[ht]
\centering
%\includegraphics[width=\textwidth]{FIG_03BW.pdf}
%\includegraphics[width=\textwidth]{FIG_03BW_LR.pdf}
\includegraphics[width=\textwidth]{FIG_03.pdf}
%\includegraphics[width=\textwidth]{FIG_03_LR.pdf}
\caption{FORC diagrams with increasing vortex effects. SF=4 for all diagrams. a) 50$\nm$; b) 60$\nm$; c) 66$\nm$; and d) 76$\nm$. At these sizes, an ever larger fraction of the particle moments begin to switch with nonuniform magnetisations, i.e., vortex nucleation. At 76$\nm$ all particles are in the single vortex remanent state. Dashed contour lines denote negative $\rho$ values.}
\label{FIG_03}
\end{figure}

\end{document}
